\documentclass{article}
\usepackage[english]{babel}
\usepackage{xltxtra}
\usepackage{hyperref}
\begin{document}
\title{Artificial Intelligence II - Homework 1 \\ Missionaries and cannibals problem}
\author{Jonáš Petrovský\\
        Faculty of Business and Economics,\\
		Mendelu University in Brno\\
		Czech Republic \\
		\texttt{jond@post.cz}}  %\texttt formats the text to a typewriter style font
\date{\today}  %\today is replaced with the current date
\maketitle

\section{Introduction}
The goal of the work is to implement an own solution to the ''Missionaries and cannibals problem''. You can see the problem description here: \url{https://en.wikipedia.org/wiki/Missionaries_and_cannibals_problem#The_problem}

\section{Solution design}
The problem can by solved by creating and searching a state space (tree). A single state has the following form:
((a1,b1,c1), (a2,b2,c2)), where a1, b1, c1 are numbers of missionaries, cannibals and boats on the left bank, and a2, b2, c2 is the same for the right bank.    

We need a starting and a target state, operators (actions) and a criterion for finding the cheapest (shortest) path.

We can use Breadth-first search (BFS) or Depth-first search (DFS). BFS finds the optimal solution among all states in the state tree, while DFS finds the first available solution (which doesn't have to be optimal).  

\section{Implementation}
The program was implemented in Python. There is a main class \texttt{MissionaryWorld}, which in its constructor accepts number of cannibals (same as missionaries) for choosing whether to solve 2+2, 3+3 or 4+4 problem. 

The state tree nodes are saved in \verb|state_tree['nodes']| variable under given ID and have a following format (example for root node): \\
\verb|root_node = {'left': self.start_state[0], 'right': self.start_state[1]| \\
\verb| 'id': 0,  'parent_id': None, 'level': 0, 'used_op': None}|

 The main method \texttt{generate\_tree} accepts a search type (DFS or BFS) as its parameter and finds the desired solution. 
\begin{enumerate}
\item Create a data structure for nodes (queue for BFS and stack for DFS)
\item Insert root node (starting state)
\item While the data structure is not empty:
	\begin{enumerate}
	\item Generate new possible states from the examined node: \verb|_generate_tree_leve|
	\item Add new nodes to the tree. If the new node is the targe state, end the program and show the solution. 
	\item Only for DFS -- check if the level is not too deep. If it is, do not add more children.
	\item Add new nodes to the data structure. 
	\end{enumerate}
\end{enumerate}

A method \verb|_check_new_state(current_state, operator)| is called inside the \verb|_generate_tree_level(current_state)| method and checks for all operators if the generated state is possible -- does not violate any conditions: enough people for transport, no more cannibals than missionaries on both banks and that the new state is not the same as the previous state of the current state.

Finally a method \verb|_show_solution(target_node)| reconstructs the path and counts total number of steps. Format description: \\ \verb|<level n.> (<node ID>): <state left>, <state right> <- <used operator>|

\section{Results}
The program can be run by typing the following commond into the operating system terminal (Python 2.7 has to be installed on the system): \\ 
\verb|python run.py <number of cannibals> <search type>| 
\\ The first parameter can be 2, 3 or 4 and the second BFS or DFS.

\subsection{3+3}
The original problem of 3 missionaries and 3 cannibals can be solved with no problems. Both BFS and DFS found the optimal solution.
\begin{itemize}
\item path price (number of river crossings): 11
\item BFS -- target state was the 55. examined state.
\item DFS -- target state was the 15. examined state.
\end{itemize}
For this problem type is better DFS because of lower number of examined states needed.

\subsection{2+2}
The edited problem was also solved easily.
\begin{itemize}
\item path price (number of river crossings): 5
\item BFS -- target state was the 20. examined state.
\item DFS -- target state was the 15. examined state.
\end{itemize}
For this problem type is better DFS because of lower number of examined states needed.




\end{document}  %End of document.