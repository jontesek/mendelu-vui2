\documentclass{article}
\usepackage[english]{babel}
\usepackage{xltxtra}
\usepackage{hyperref}
\usepackage{listings}
\begin{document}
\title{Artificial Intelligence II - Homework 2 \\ 8-puzzle}
\author{Jonáš Petrovský\\
        Faculty of Business and Economics,\\
		Mendel University in Brno,\\
		Czech Republic \\
		\texttt{jond@post.cz}}
\date{\today}
\maketitle

\section{Introduction}
The goal of the work is to implement a solution to the ''8-puzzle'' using two assigned heuristics. There is a board 3x3 with 8 tiles inside (and one blank square). The goal is to rearrange all tiles so they are in order. 

\section{Solution design}
The 8-puzzle can by solved by creating and searching a state space. It has size of 9! = 362,880 states. Only half of the states are reachable from any given (start) state -- are solvable (Latombe, 2011). The number is quite large and for 15-puzzle (16! possible configurations) it's much higher. 

But another way of solving the puzzle exists -- to use a heuristic. This way, the optimal (or even any) solution may not be found, but the process is much faster. Following two heuristics are used: h1 -- min. wrong placings, h2 -- min. Manhattan distance.

A board state is represented by a hash of <tile name> => <tile x,y coordinates>, ''moved tile'' field and `'h\_value'' field. Tiles are named 1--8, a blank square is named 0 and behaves like a regular tile. 

\section{Implementation}
The program was implemented in Python. A state is saved in a variable with the following structure (example for goal state).
\begin{lstlisting}[language=python]
self.goal_state = {
            'tiles': {
                1: (1,1), 2: (1,2), 3: (1,3),
                8: (2,1), 0: (2,2), 4: (2,3),
                7: (3,1), 6: (3,2), 5: (3,3)
            },
            'moved_tile': None,
            'h_value': 0
}
\end{lstlisting}
All visited states are saved in ''solution\_path'' list (in order). There is a main class \texttt{PuzzleWorld} with 4 public methods: 
\begin{itemize}
\item \verb|solve_given_puzzle(start_state, h_type)|
\item \verb|solve_random_puzzle(h_type)|
\item \verb|bulk_solve(number_of_puzzles, h_type)|
\item \verb|bulk_solve_compare(number_of_puzzles)|
\end{itemize}

The main method \verb|solve_given_puzzle| accepts a start state and tries to find a solution using given heuristic. The general algorithm:
\begin{enumerate}
\item Reset solution path.
\item Insert a start state into solution path.
\item While tiles in current state != tiles in goal state:
	\begin{enumerate}
	\item Find possible moves towards zero tile.
           \item For every move create a new state.
           \item Check if the states are valid -- so far unvisited (not present in solution path). If there are no valid states, a solution cannot be found -- a message of failure is displayed and the program ends.
           \item Calculate heuristic value for every new valid state.
           \item Choose a state with the lowest heuristic value, insert it into solution path and mark it as current.
           \item Return solution path.
	\end{enumerate}
\end{enumerate}

If the goal state was reached, a message of success and number of performed moves (length of \texttt{solution\_path} - 1) are displayed. 

The class \texttt{PuzzleHeuristics} implements the calculation of heuristics:
\begin{enumerate}
\item Hamming distance -- number of tiles out of place.
\item Manhattan distance -- the sum of minimum number of steps (vertical and horizontal) to move every tile in its goal position.
\end{enumerate}
For both heuristics we choose the minimal value. 

\section{Results}
The program can be run by typing the following command into the operating system's terminal (Python 2.7 has to be installed on the system): \\ 
\verb|python run-2.py| \\
What the program should do must be stated in the run-2.py source code (the script has no parameters).

\subsection{Hamming distance}



\section{References}

Latombe, J.C. Search problems. \textit{Stanford AI Lab} [online]. 2011 [accessed 8.11.2015]. Available at: \url{http://ai.stanford.edu/~latombe/cs121/2011/slides/B-search-problems.ppt}


\end{document}  %End of document.